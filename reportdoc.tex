\documentclass[12pt,svgnames,oneside]{book}

% URLs and hyperlinks ---------------------------------------
\usepackage{hyperref}
\hypersetup{
colorlinks=true,
linkcolor=Navy,
filecolor=magenta,
urlcolor=Navy,
}
\urlstyle{same}
%------------------------------------------------------------

% resourses -------------------------------------------------
\usepackage{natbib}
%------------------------------------------------------------

% tables ----------------------------------------------------
\usepackage{float}
\usepackage{longtable}
\usepackage{rotating}
\renewcommand{\arraystretch}{1.4}
%------------------------------------------------------------

% section numbering -----------------------------------------
\setcounter{secnumdepth}{3}
\setcounter{tocdepth}{3}
%------------------------------------------------------------

% titlepage -------------------------------------------------
\usepackage{pdfpages}
%------------------------------------------------------------

% checkmark -------------------------------------------------
\usepackage{amssymb}
%------------------------------------------------------------

% inline lists ----------------------------------------------
\usepackage[inline]{enumitem}
% -----------------------------------------------------------

\usepackage{adjustbox}

% persian support -------------------------------------------
\usepackage{xepersian}
\settextfont{Yas}
%------------------------------------------------------------

\newcounter{itemadded}
\setcounter{itemadded}{0}


\newcommand{\addeditem}{%
\addtocounter{enumi}{-1}%
\stepcounter{itemadded}
\let\LaTeXStandardLabelEnumi\labelenumi%
\addtocounter{enumi}{1}
\renewcommand{\labelenumi}{\arabic{enumi}\lr{R}.}%
\item 
% Switch back to old labelling 
\let\labelenumi\LaTeXStandardLabelEnumi%
}%


\let\LaTeXStandardEnumerateBegin\enumerate
\let\LaTeXStandardEnumerateEnd\endenumerate

\renewenvironment{enumerate}{%
\LaTeXStandardEnumerateBegin%
\setcounter{itemadded}{0}
}{%
\LaTeXStandardEnumerateEnd%
}%

% commands --------------------------------------------------
\newcommand{\uc}[1]{\lr{U{#1}}}
\newcommand{\req}[1]{\lr{R{#1}}}
\newcommand{\tucbw}{\lr{TUCBW}}
\newcommand{\tucew}{\lr{TUCEW}}
\newcommand{\tuc}[2]{
\begin{itemize}
\item[] \tucbw: {#1}
\item[] \tucew: {#2}
\end{itemize}}

\newcommand{\zstar}{$\star$}
\newcommand{\actorsystem}[1]{
کنشگر: {#1} &
سیستم: کارتاپ \\
}
\newcommand{\zerostep}[1]{
& 0. {#1} \\
}
% -----------------------------------------------------------

% Counters --------------------------------------------------
\newcounter{UseCaseCounter}
\newcommand{\step}[1]{
\stepcounter{UseCaseCounter}\arabic{UseCaseCounter}. {#1}
}
\newcommand{\ucname}[2]{
\multicolumn{2}{|r|}{\uc{0{#1}}: {#2}} \\
}
\newcommand{\preif}[1]{
\multicolumn{2}{|r|}{پیش‌شرط:‌ {#1}} \\
}

\newcounter{UseCaseListCounter}
% \ifnum\value{page}<10 0\fi\arabic{page}
\newcommand{\ucstep}[1]{
\lr{U\stepcounter{UseCaseListCounter}\arabic{UseCaseListCounter}}: {#1}
}
% -----------------------------------------------------------
\begin{document}
\renewcommand{\bibname}{مراجع}
\includepdf{title}
\frontmatter
\tableofcontents
\mainmatter

\chapter{سند نیازمندی‌ها}

\section{مقدمه}	
با توجه به افزایش روز افزون نرخ بیکاری در کشور ما کاریابی به صورت چشم‌گیر مورد توجه تمامی اقشار جامعه قرار گرفته است. بدین منظور ایجاد یک سامانه هدفمند برای کاهش این نرخ، سودمند است. سامانه نرم افزاری \textbf{کارتاپ}، با معرفی کارجویان به کارفرمایان و توانمندسازی افراد به منظور دریافت کار، این نیاز مهم را برآورده می سازد.

\subsection{هدف}
یکی از بزرگ‌ترین نیازهای جامعه امروز، یافتن شغل مناسب برای افراد است. در گذشته‌ای نه چندان دور، کارجویان برای پیدا کردن شغل، باید به دفاتر کاریابی مراجعه می‌کردند؛ اما مدتی‌ست که دیگر هر کاری از طریق اینترنت و به صورت آنلاین صورت می‌گیرد. با توجه به رقابت زیاد و اینترنتی شدن تمام امور، بهترین راه برای رفع این نیاز، طراحی پلتفرم کاریابی‌‌ای است که فضایی برای کارفرمایان و کارجویان فراهم می آورد تا بتوانند به راحتی به هدف خود برسند.
سامانه‌ی کاریابی به این صورت است که مشاغل را در دسته‌بندی‌های متفاوتی به کاربر نمایش می‌دهد و با استفاده از فیلترها، کارجویان میتوانند لیست مشاغل مد نظر خود را بیابند. همچنین برای سهولت کاربران امکان ساخت رزومه با قالب‌های حرفه‌ای و آماده را برای کارجویان فراهم می‌کند. کارفرما‌ها می‌توانند با پرداخت مبلغی، آگهی خود را روی سامانه قرار دهند تا به افراد جویای کار نمایش داده شود. همچنین کارفرماها می‌توانند مهارت‌های مورد نیاز برای موقعیت شغلی مورد نظر و همچنین، نوع کار از لحاظ پاره‌وقت، تمام‌وقت ، دورکاری و... را مشخص کنند.				علاوه بر موارد فوق این کار باعث شده تا نرخ بیکاری در کشور کاهش پیدا کند و افراد در کوتاه ترین زمان بتوانند شغل مورد نظر خود را پیدا کنند.

\subsection{قلمرو}			
این محصول که به نام کارتاپ شناخته می‌شود، بستری است که در آن متقاضیان کار می‌توانند شغل متناسب با مهارت‌های خود را جست‌وجو کنند و موقعیت‌های کاری مختلف را مقایسه کنند.
در کنار این موارد، بخش مهارت افزایی نیز وجود دارد که افراد می‌توانند با کسب آموزش‌های مورد نظر و کسب گواهی معتبر، خود را برای موقعیت‌های شغلی مختلف آماده کنند.

\subsection{تعاریف، سرنام‌ها و کوته نوشته‌ها}		
به جدول \ref{words} مراجعه شود.

\begin{sidewaystable}
\begin{center}
\caption{جدول واژگان، سرنام‌ها و کوته‌نوشته‌ها}
\begin{tabular}{|c|c|p{9cm}|}

\hline
واژه &
\centering واژه‌ی کامل &
توضیح \\
\hline
\hline
\lr{GPS} &

\lr{Global Positioning System} &
سامانه‌ای برای یافتن موقعیت جغرافیایی است. \\ 
\hline

\lr{HTTPS} & \lr{Hypertext Transfer Protocol Secure} &
به معنای پروتکل انتقال ابر متنی است و وظیفه‌ی ‌ارسال و دریافت داده‌ها بین کلاینت و سرور را بر عهده دارد.\\ 
\hline

\lr{HTML} & \lr{Hypertext Markup Language} &
زبان استایل دهی و ویرایش ویژگی‌های ظاهری محتوای صفحات وب می‌باشد. \\ 
\hline

\lr{CSS} & \lr{Cascading Style Sheets} & 
زبان استایل دهی و ویرایش ویژگی‌های ظاهری محتوای صفحات وب می‌باشد. \\ 
\hline

\lr{SRS} & \lr{Software Requirement Specification} &
 به معنی مشخصات مورد نیاز نرم افزار می‌باشد.\\ 
\hline

\lr{CPU} & \lr{Central Processing Unit} &
 به معنی واحد پردازش مرکزی می‌باشد. \\ 
\hline

\lr{RAM} & \lr{Random Access Memory} &
 نوعی از حافظه‌ی کامپیوتری است که به هر ترتیبی قابل خواندن و تغییر است. \\ 
\hline

\lr{SSL} & \lr{Secure Sockets Layer} &
 فناوری امنیتی استاندارد برای برقراری یک پیوند رمزگذاری شده بین یک سرور و یک سرویس گیرنده می‌باشد. \\ 
\hline

\lr{PDF} & \lr{Portable Document Format} &
 فایل‌هایی هستند که برای باز کردن در وسائل مختلف به منظور مطالعه‌ی متن یا پرینت کردن آن به کار می‌روند. \\ 
\hline

\lr{SSD} & \lr{Solid State Drive} &
 به معنی درایو حالت جامد می‌باشد \\ 
\hline

\lr{UI} & \lr{User Interface} &
 نوعی فضایی است که تعامل میان انسان و ماشین در آن رخ می‌دهد \\ 
\hline

\lr{UX} & \lr{User Experience} &
یک طراحی کاربر محور به این معناست که شما باید محصول یا خدماتی را ارائه بدهید که دقیقا همانکاری را انجام بدهد که کاربر می‌خواهد.  \\ 
\hline

\lr{JavaScript} & &
یک زبان برنامه نویسی می‌باشد که به وسیله‌ی آن می توان بین کاربر و سایت ارتباط برقرار نمود. \\ 
\hline

کارجو & &
شخصی است که به دنبال فرصت شغلی و کار می‌باشد. \\
\hline

کارفرما & &
شخصی است که به علت نیاز نیروی انسانی در شرکتش، کارجویان را با توجه به مهارتشان و نیاز شرکتش، استخدام می‌کند \\
\hline
\end{tabular}\label{words}
\end{center}
\end{sidewaystable}

\subsection{مراجع}			
برای بررسی مرجع استفاده شده به
\cite{kung2013object}				
مراجعه کنید.

\subsection{طرح کلی}		
روند کار در سند تدوین شده به این صورت است که در ابتدا اهداف و ویژگی های محصول شرح داده می‌شود و سپس به واسط‌های مختلف (من جمله واسط‌های سیستم، کاربر، سخت‌افزاری،نرم‌افزاری و...)، کارکردهای محصول ،مشخصات کاربران سیستم، قیود، مفروضات و وابستگی‌ها پرداخته و در نهایت به نیازمندی‌های آن خواهیم پرداخت.

\section{شرح کلی}
کارتاپ یک سیستم نرم‌افزاری برای کاریابی هدفمند در سازمان‌ها و شرکت‌هاست.
از طریق این سامانه، کارفرما نیاز‌های استخدامی خود را مطرح نموده و سپس بر اساس شغل و قابلیت‌های اعلام شده، بایستی بتواند به طور هوشمندانه کارجویان مناسب را به وی معرفی نماید. به نحوی می‌توان گفت این سیستم به منظور هوشمندسازی حداکثری روال‌های سنتی در این زمینه است.
از جمله امکانات این سیستم می‌توان به امکان ثبت نام کرفرما، ثبت اطلاعات شرکتی، اعلام نیاز استخدامی، ثبت آگهی و همچنین برای کارجویان، ایجاد پروفایل و رزومه شخصی اشاره نمود.

\subsection{چشم‌انداز محصول}
بر اساس سیستم مذکور درخواست‌های مورد نیاز برای کاربران با توجه به خواسته ارسال می‌شود و آن‌ها می‌توانند با بررسی درخواست‌ها و فایل‌های پیوست نظرات خود را اقدام کرده و در صورت نیاز با یک‌دیگر ارتباط بگیرند.
از جمله امکانات این سیستم دریافت رزومه، درخواست اخذ تست‌های بالینی برای کارفرمایان و همچنین شرکت در تست‌های شخصیت شناسی، ساخت رزومه شخصی، انتخاب علایق شغلی برای کارجویان اشاره کرد.

\subsubsection{واسط‌های سیستم}
واسط‌های سیستم این مسئله را بیان می‌کند که ارتباط سامانه‌ی ما با سیستم‌های خارجی، از طریق چه واسطه‌هایی صورت می گیرد و چگونه با هم در تبادل اطلاعات مختلف هستند. به عنوان مثال:
\begin{enumerate}
\item
	دسترسی به پایگاه‌داده‌ی اداره‌ی ثبت احوال برای احراز هویت کارجو‌یان، مورد نیاز است.
\item
	دسترسی به پایگاه‌داده‌ی اداره‌ی ثبت شرکت‌ها برای احراز هویت شرکت‌ها، مورد نیاز است.
\item
	از آنجایی که این پلتفرم کاربران زیادی خواهد داشت، به سرور‌های قدرتمند و سریعی جهت پاسخ به درخواست‌ها و انجام عملیات‌های لازم، نیاز داریم.
\item
	جهت ارتباط و اطلاع رسانی‌های مهم به کاربران از طریق پیامک، نیاز به ارتباط با سازمان‌های مخابراتی یا شرکت‌هایی‌ست که این نوع خدمات را ارائه می دهند.
\end{enumerate}

\subsubsection{واسط‌های کاربر}
جهت استفاده‌ی صحیح و کارآمد کاربران از سامانه، یک سری قابلیت‌های عمومی برای همگان و یک سری قابلیت‌های خاص در پنل کاربری کاربرانِ وارد شده در حساب کاربری، وجود دارد. در نتیجه نقش کاربران تعیین کننده‌ی سطح دسترسی آن‌ها می‌باشد. سطح‌ دسترسی یا نقش کاربران در این سامانه، به دو دسته تقسیم می شود:
\begin{enumerate}
\item
 کارفرما
\item
 کارجو
\end{enumerate}
\subsubsection{واسط‌های سخت‌افزاری}
واضح است سیستم نرم‌افزاری کاریابی، نیازهای سخت‌افزاری به‌خصوصی ندارد؛ با این وجود لیستی از واسط های سخت‌افزاری مورد نیاز اولیه در ادامه آمده است:
\begin{enumerate}
\item
ابزارهای اولیه جهت پردازش و مدیریت داده‌ها و عملیات:
\begin{itemize}
\item
کارت شبکه
\item
مودم (اتصال اینترنت)
\item
سرور شبکه
\item
سرور پردازش داده
\end{itemize}

\item
ابزار لازم برای پیدا کردن مکان دقیق شرکت‌ها:
\begin{itemize}
\item
سرویس \lr{GPS}
\end{itemize}

\item
دستگاه‌های موردنیاز جهت ارتباط افراد با بستر اینترنت (هر سخت‌افرازی که توانایی اجرای نرم‌افزارهایی نظیر مرورگرها را داشته باشد) مانند:
\begin{itemize}
\item
تلفن همراه
\item
کامپیوتر شخصی
\item
تبلت
\item
لپ‌تاپ
\end{itemize}

\end{enumerate}
\subsubsection{واسط‌های نرم‌افزاری}\label{software}
\begin{itemize}
\item
مرورگر‌های مرسوم همچون
\lr{Google Chrome}،
\lr{Mozilla Firefox} و
\lr{Microsoft Edge}						که از آخرین نسخه‌های
\lr{HTML}،
\lr{CSS}						و
\lr{JavaScript}						پشتیبانی می‌کنند.

\item
با توجه به حجم بالای داده‌ها، استفاده از سیستم‌های پایگاه‌ داده‌ی رابطه‌ای
\LTRfootnote{Relational databases}
و پایگاه‌داده‌های غیر رابطه‌ای
\LTRfootnote{NOSQL databases}
\item
هر نرم‌افزاری که بتواند فایل با فرمت \lr{PDF} را نشان بدهد.
\end{itemize}
\subsubsection{واسط‌های ارتباطی}
این سیستم به صورت تحت‌ وب است که کاربران با توجه به نیاز‌ها با سرور و پایگاه داده ثبت احوال و اداره ثبت شرکت‌ها ارتباط گرفته تا احراز هویت شوند و کار مورد نظر خود را انجام دهند.

\subsubsection{واسط‌های حافظه}
از آنجا که در سیستم، لازم است اطلاعات ضروری کاربران که بخش اعظم جامعه را تشکیل می‌دهند، ذخیره و آمارگیری‌های مورد نیاز از طریق این داده‌ها استخراج شود، پس منطقی است که حافظه‌ی جانبی قابل توجهی به سیستم اختصاص یابد. همچنین در					پروسه‌ی تخصیص حافظه، نیاز سیستم به پردازش سریع داده‌ها در مراحل جستجو میان مشاغل در نظر گرفته شده ‌است.					پس به طور کلی:

\begin{enumerate}
\item
باتوجه به حجم پردازشی بالای این وب‌سایت جهت انجام امور مختلف، این سامانه نیازمند \lr{CPU}های قدرتمند و به‌روز و همچنین حافظه‌های عظیم و پرسرعت (همانند \lr{SSD}) است.

\item
همچنین از \lr{RAM}های قدرتمندی برای تسریع درخواست ها استفاده می‌شود.
\end{enumerate}
\subsubsection{واسط‌های عملیات}
\begin{enumerate}
\item
اطلاعات بین سامانه و پایگاه داده، به صورت خودکار تبادل می شود و به صورت دستی چیزی تغییر نمی‌یابد (مگر در صورت ایجاد مشکلی خاص.)
\item
برای این سامانه، نیاز به سرورهای قدرتمند و سریعی برای پردازش و ذخیره سازی داده‌ها نیاز است.
\item
مراحل اعتبارسنجیِ صحت اطلاعات ورودی و فیلترهای جست‌و‌جو به صورت خودکار، توسط سامانه انجام می‌شود.
\item
تمامی اطلاعات ویرایش شده یا بارگذاری شده، در همان لحظه
 صورت \lr{real time} \RTLfootnote{به سیستم‌‌‌هایی گفته می‌شود که به صورت بی‌درنگ و بدون نیاز به بارگذاری (\lr{reload}) مجدد صفحه‌، اطلاعات بروزشده نمایش داده می‌شوند؛ پیام‌رسان‌ تلگرام از بهترین مثال‌های این سیستم‌هاست.})	 در سرور‌های سامانه بروزرسانی یا بارگذاری می‌شوند.
\item
در صورت استفاده‌ی بیش از حد مجاز تعداد کاربران جهت متعادل سازی سامانه، باید از طریق هدایت ترافیک به چندین سرور، دسترسی به یک دامنه را آسان‌تر و سریع‌تر کرد.
\item
ارسال پیامک‌های انبوه به کاربران جهت اطلاع رسانی‌های مهم، به طور خودکار توسط سیستم‌های ارائه دهنده‌ی این نوع خدمات، انجام می‌شود.
\item
سامانه باید به صورت خودکار رزومه‌های کارجویان را با درخواست‌های شغلی کارفرمایان مقایسه کند و در صورت مطابقت به طرفین پیشنهاد دهد.
\item
سامانه باید مهارت‌های کارجویان را از رزومه‌های آن‌ها به طور خودکار استخراج کند.
\item
احراز هویت شرکت‌ها به صورت خودکار انجام شود.
\end{enumerate}

\subsubsection{نیازمندی‌های سازگاری با محیط نصب}
این سامانه روی تمامی دستگاه‌هایی که دارای مرورگر مورد نیاز در \ref{software} اشاره شده است، قابل اجرا می‌باشد و نیازی به نصب ندارد.

\subsection{کارکرد محصول}
این سیستم که به منظور سهولت در روند استخدام افراد در شرکت‌ها و یا پیدا کردن شغل توسط کارجویان طراحی شده‌ است، دارای قابلیت‌های متنوع برای هرکاربر می‌باشد:
\begin{enumerate}
\item
کارجویان
\begin{itemize}
\item
کشف فرصت‌های شغلی
\item
معرفی شرکت‌ها و فرصت‌های شغلی موجود در هرکدام
\item
آگاهی از مشاغل جدید
\item
استفاده از فیلتر های پیشرفته برای یافتن مهارت، نوع ساعت کاری
\item
رزومه ساز آنلاین با قالب های پیشرفته و حرفه‌ای
\item
ارتباط آسان با کارفرمایان
\item
افزایش  مهارت‌های فردی کارجویان برای پیدا کردن شغل بهتر
\item
آموزش قوانین حقوقی به کارجویان برای جلوگیری هرچه بیشتر از کلاهبرداری‌های اینترنتی و شغلی
\end{itemize}

\item
کارفرمایان
\begin{itemize}
\item
جذب نیرو و درج آگهی استخدام
\item
امکان تحلیل و بهینه‌سازی آگهی با استفاده از آمار دقیق.
\item
مدیریت رزومه‌های دریافتی در پنل شرکت
\item
مدیریت وضعیت درخواست متقاضی از داخل سیستم و اطلاع‌دهی به کارجو.
\item
معرفی و تبلیغ برند
\item
جستجو در رزومه‌های دریافتی
\item
یادداشت گذاری بر روی رزومه‌ها
\item
انتشار رایگان آگهی‌ کارآموزی
\end{itemize}

\end{enumerate}
از دیگر قابلیت‌های سیستم به موارد زیر میتوان اشاره کرد:
\begin{itemize}
\item
بخش مقالات و اخبار برای افزایش اطلاعات کاربران
\item
همگام با اصول بهینه سازی برای موتورهای جستجو
\end{itemize}

\subsection{مشخصات کاربر}
کاربران کارتاپ به دو دسته‌ی کارفرمایان و کارجویان تقسیم می شوند:

\begin{enumerate}
\item
کارجویان
این دسته از کاربران شامل افرادی از جامعه هستند که در جست‌وجوی کاری مطابق با مهارت‌ها، استعدادها و یا مدرک تحصیلی خود با توجه به شرایطی همچون محل اقامت، میزان ساعات کاری و... می‌باشد. از این دسته افراد انتظار می‌رود که علاوه بر دسترسی به اینترنت، توانایی کار با مرورگر، ثبت نام، بارگذاری یا تشکیل رزومه، احراز هویت و همچنین آشنایی با زبان فارسی را داشته باشند.
\item
کافرمایان

این دسته از کاربران شامل افراد یا شرکت‌هایی هستند که در صدد پذیرش یا استخدام کارجو می‌باشند. آنها پس از بررسی و پذیرش رزومه‌ی کارجویان، مهارت‌ها و شرایط موردنظر خود را با مشخصات کارجو سنجیده و در صورت تطابق، کارجو را استخدام می‌کنند. این دسته از کاربران علاوه بر انتظاراتی که از کارجویان می‌رود ،ملزم به دارا بودن کد ثبت شده‌ی شرکت و پروانه‌ی کسب نیز می‌باشند.
\end{enumerate}

\subsection{قیود}
\begin{enumerate}
\item
دسترسی به کارتاپ باید به صورت شبانه‌روزی برای کاربران فراهم باشد.
\item
واسط‌های کاربری کارتاپ باید شرایط آسان و قابل‌فهمی را برای کاربران فراهم سازد.
\item
کارتاپ باید در کمتر از ۱۸ ماه به مشتری تحویل داده شود.
\item
هزینه تحلیل، طراحی و توسعه ی کارتاپ مطابق بودجه پروژه باید حداکثر \lr{50,000,000,000} ریال باشد.
\end{enumerate}
\subsection{قوانین کسب‌و‌کار}
\begin{itemize}
\item
رمز شخصی به هنگام احراز هویت و رمز موقت برای هر بار ورود، به شماره تلفن همراهی که کاربر هنگام ثبت نام وارد میکند فرستاده می‌شود.
\item
با توجه به اجباری بودن بیمه، کارفرمایان موظف هستند که شرایط بیمه کردن کارجویان را فراهم سازند.
\item
استخدام کارجویان توسط کارفرمایان در چارچوب قوانین اداره کار صورت می‌پذیرد.
\item
هر کارفرما برای ثبت شرکت باید دارای کد تایید شده توسط سامانه ثبت شرکت‌ها باشد.

\end{itemize}
\subsection{مفروضات و وابستگی‌ها}
در این قسمت هر یک از عوامل موثر بر الزامات مندرج در \lr{SRS} که می‌توانند بر آن تأثیر بگذارند، آورده شده است:

\begin{enumerate}
\item
وابستگی‌ها
\begin{itemize}
\item
به دلیل حجم بالای اطلاعات، سیستم به پایگاه داده‌های کلان داده وابسته است.
\item
اطلاعات پایگاه داده‌های اداره ثبت شرکت‌ها در جریان‌های کاری سیستم، مورد نیاز است.
\item
جهت ارتباط و اطلاع رسانی‌های مهم به کاربران از طریق پیامک نیاز به ارتباط با سازمان‌های مخابراتی یا شرکت‌هایی است که این نوع خدمات را ارائه می‌دهند.
\end{itemize}

\item
مفروضات
\begin{itemize}
\item
کاربر توانایی دسترسی به اینترنت و تسلط کار با آن را داشته باشد.
\item
کاربر از دستگاهی با قابلیت اتصال به اینترنت و اجرای مرورگر جهت استفاده از خدمات سامانه، برخوردار است.
\item
کاربر حداقل دانش مورد نیاز برای کار با دستگاه‌های هوشمند را دارد.
\item
مرورگر کاربر از جاوا اسکریپت پشتیبانی کند.
\end{itemize}
\end{enumerate}

\section{نیازمندی‌های خاص}

\subsection{نیازمندی‌های واسط خارجی}
\begin{enumerate}
\item
سیستم داده‌هایی را از ثبت احوال می‌گیرد و پس از آن کارجویان را  احراز هویت می‌کند.
\item
سیستم کد مربوط به هر شرکت را، به اداره ثبت شرکت‌ها می‌فرستد و جواب احراز هویت شرکت‌ها را دریافت می‌کند.
\item
سیستم با ارتباط با سازمان‌های مخابراتی و شرکت‌های اپراتور همراه‌اول، ایرانسل و یا رایتل به کاربران پیامک‌هایی با موضوعاتی از قبیل ارسال کدتایید، اطلاع‌رسانی، اخبار و ... می‌فرستد.
\end{enumerate}

\subsection{نیازمندی‌های کارکردی}
برای فهم راحت‌تر و چیدمان بهتر، نیازمندی‌ها به سه دسته‌ی پلتفرم، کارجو و کارفرما تقسیم شده‌اند.
\RTLfootnote{این تقسیم‌بندی قرار نیست خیلی دقیق باشد، چون مفهوم مطالب در بعضی موارد خیلی بهم نزدیک هستند؛ این کار صرفا برای جداسازی موارد مشابه بهم صورت گرفته است.}

\begin{enumerate}

\addeditem
کارتاپ باید امکان ثبت درخواست برای آگهی‌های شغلی متفاوت را برای کارجو فراهم سازد.

\begin{enumerate}
\item[1.1\lr{.R}]
کارتاپ باید به هنگام ثبت درخواست کارجو، امکان وارد کردن حقوق پیشنهادی وی را فراهم کند
\end{enumerate}

\addeditem
کارتاپ باید امکان نشاندار کردن و ذخیره کردن آگهی‌ها را برای کارجویان فراهم سازد.

\addeditem
کارتاپ باید آگهی‌های پیشنهادی مطابق با اطلاعات کارجو را نمایش دهد. 

\addeditem
کارتاپ باید قسمتی را به عنوان صفحه شخصی کارجو شامل پروفایل، اطلاعات شخصی، علایق و دسته‌بندی مشاغل داشته باشد.

\addeditem
کارتاپ باید امکان تغییر مشخصات شناسنامه‌ای، اطلاعات تماس و محل اقامت را داشته باشد.

\addeditem
کارتاپ باید قسمتی را به عنوان پنل کاربری برای نمایش آخرین وضعیت و روند تمامی درخواست‌ها، شامل:

\begin{itemize}
\item
ارسال شده
\item
در حال بررسی
\item
دیده شده توسط کارفرما
\item
تایید یا رد درخواست
\item
علل تایید یا رد درخواست
\end{itemize}
را اختصاص دهد.

\addeditem
کارتاپ باید توانایی ایجاد و تشکیل رزومه‌ی الکترونیکی (رزومه ساز) برای کارجویان را فراهم نماید.

\addeditem
کارتاپ باید قابلیت بارگذاری فایل رزومه را برای کارجویان فراهم نماید. 

\addeditem
کارتاپ باید قسمتی را برای نمایش روند تمامی پیشنهادهای دیگر کارفرمایان برای استخدام کارجو اختصاص دهد. 

\addeditem
کارتاپ باید آگهی‌های فوری و آگهی‌های پیشنهادی را برای کارجو نمایش دهد.


\addeditem
کارتاپ باید امکان فیلتر کردن آگهی ها بر حسب زمان نشر آنها و همچنین  مواردی از قبیل نام استان و شهر، نوع مهارت‌ها و انتخاب نوع موقعیت شغلی را برای کارجویان فراهم سازد.

\addeditem
کارتاپ باید امکان فرستادن رزومه به چندین آگهی به صورت همزمان را داشته باشد. 


\addeditem
کارفرما باید امکان ثبت آگهی شغلی را در این سیستم داشته باشد.

\addeditem
کارتاپ باید امکان ثبت‌نام شرکت‌ها را براحتی در اختیار کارفرما‌یان قرار دهد.

\addeditem
کارتاپ باید امکان بارگذاری تصاویری از محیط کاری،فضای شرکت و... را برای کارفرمایان فراهم کند. 

\addeditem
کارتاپ باید امکان بارگذاری موقعیت مکانی شرکت توسط کارفرما را فراهم سازد.

\addeditem
کارتاپ باید بتواند کارجویان مناسب و مطابق با شرایط آگهی‌های شرکت‌ها را یافته و آنان را به کارفرما‌ها پیشنهاد دهد.

\addeditem
کارتاپ باید امکان وارد کردن اطلاعاتی نظیر شرایط کاری، دستمزد، جنسیت و انتظارات عمومی و تخصصی از سوی کارفرما را فراهم کند.

\addeditem
کارتاپ باید یک صفحه مربوط به اطلاعات شرکت، پرسنل شرکت، آگهی‌های فعال، آگهی‌های منقضی شده، تصاویر، درخواست‌های کارجویان و پیشنهاد‌های ارائه شده به کارجویان برتر را به طور کامل نمایش دهد.

\addeditem
کارتاپ باید امکان ایجاد اکانت پرمیوم و خرید اشتراک برای کارفرمایان جهت ثبت بیش از 10 آگهی و همچنین ایجاد دیگر امکانات را فراهم کند.

\addeditem
کارتاپ باید برای ثبت نام کارجویان، اطلاعاتی را از قبیل نام و نام‌خانوادگی، تلفن همراه و ایمیل را از کاربر دریافت نماید.

\addeditem
کارتاپ باید هنگام ثبت درخواست کارجو، عملیات احراز هویت کارجو (دریافت کد ملی و بررسی صحت آن، فرستادن کد تایید موقت برای تایید شماره تلفن) را فراهم کند

\addeditem
کارتاپ باید امکان ورود به سامانه را برای کاربران فراهم سازد.

\begin{enumerate}
\item[1.23\lr{.R}]
کارتاپ باید امکان بازیابی رمز عبور کاربر را در صورت فراموشی، از طریق شماره همراه و یا ایمیل ثبت شده در سامانه فراهم کند.

\item[2.23\lr{.R}]
کارتاپ باید برای هر رمز موقت، اعتبار ۱ دقیقه ای قائل شود و بعد از این زمان رمز منقضی شود.
\end{enumerate}

\addeditem
کارتاپ باید برای ایجاد آگهی استخدامی توسط کارفرما، عملیات احراز هویت، شامل: 
\begin{itemize}
\item نام شرکت
\item	 شماره‌ی ثبت شرکت یا شماره ملی شرکت
\end{itemize}
را داشته باشد.

\addeditem
سامانه باید قابلیت چت آنلاین را با کارشناس مربوطه برای کاربر فراهم نماید. 
\addeditem
کارتاپ باید امکان خارج شدن از سامانه را برای کاربر فراهم کند.

\end{enumerate}
\subsection{نیازمندی‌های کارایی}
\begin{enumerate}
\item
سامانه باید توانایی پاسخ گویی هم زمان ۱۰۰۰۰ کاربر را داشته باشد.
\item
سامانه باید برای ورود کاربران از کد \lr{CAPCHA}
\RTLfootnote{\lr{CAPCHA} یا همان کپچا، نرم‌افزاری آنلاین برای تولید سوالات و آزمون‌هایی‌ست که انسان به‌راحتی قادر به پاسخ‌گویی به آنهاست ولی کامپیوتر‌ها در حال حاضر، قادر به تشخیص و پاسخ به آنها نیستند. عبارت \lr{CAPCHA} مخفف عبارت \lr{Completely Automated Public Turing Test To Tell Computers and Humans Apart} است.}
استفاده کند تا از اینکه فرد وارد شده ربات نباشد، اطمینان حاصل کند.
\item
سامانه باید برای ثبت نام کاربران با استفاده از کد احراز هویت، هویت افراد را تایید نماید.
\item
سیستم پیامکی سامانه باید بتواند پیامک‌ها را حداکثر ظرف ۲۰ ثانیه برای کاربران ارسال کند.
\item
سامانه باید طراحی کاربرپسند داشته باشد.
\item
کارتاپ باید در هرگونه مواجه شدن با خطا، چه از سمت کاربر و چه از سمت سرور، اخطار را با جزئیات گزارش دهد، تا نیروهای فنی این مورد را در اولین زمان ممکن بازبینی و رفع کنند.

\end{enumerate}

\subsection{قیود طراحی}
\begin{enumerate}
\item
امکان بارگیری رزومه‌ها به فرمت \lr{PDF} برای کاربران فراهم باشد.
\item
سامانه باید بر روی تمامی مرورگر‌های مرسوم همچون
\lr{Google Chrome}،
\lr{Mozilla Firefox} و
\lr{Microsoft Edge}  قابل اجرا باشد.
\end{enumerate}

\subsection{صفت‌های سیستم‌ نرم‌افزاری}
\begin{enumerate}
\item امنیت
\begin{itemize}
\item
استفاده از قابلیت‌های پنل کاربری، فقط باید توسط کاربران احراز هویت شده، قابل دسترسی باشد.
\item
سامانه باید حافظ اطلاعات شخصی کاربران باشد.
\item
سامانه باید قابلیت پشتیبان‌گیری از اطلاعات سایت، که شامل اطلاعات کابران هم می‌شود و همچنین توانایی بازیابی اطلاعات را داشته باشد.
\item
به جهت افزایش و پایداری امنیت ارتباط سرور با سیستم کاربر، از پروتکل‌های امنیتی مانند \lr{SSL} و \lr{HTTPS} استفاده می‌شود.
\item
سامانه باید در صورت دریافت درخواست‌های بیش از حد مجاز اقدام به مسدود سازی کاربر به طور موقت کند.
\item
سامانه باید به طور لحظه‌ای اقدام به ذخیره‌ی اطلاعات تغییر یافته کند.
\item
سامانه باید در شرایط خاص خطاها را متوقف کند.
\end{itemize}

\item در دسترس بودن
\begin{itemize}
\item
سامانه باید به طور شبانه روز به جز بازه‌ی اصلاحات دوره‌ای، قابل دسترسی باشد.
\item
سامانه باید از طریق تمامی مرورگر‌های مرسوم مانند
\lr{Google Chrome}،
\lr{Mozilla Firefox}،
و
\lr{Microsoft Edge}
که از آخرین نسخه‌های
\lr{HTML}،
\lr{CSS}
و
\lr{JavaScript}
پشتیبانی می‌کنند، در دسترس باشند.
\item
قابلیت مشاهده‌ی آگهی‌های استخدامی، حتی در صورت عدم ورود به حساب کاربری وجود داشته باشد.
\end{itemize}

\item پشتیبانی
\begin{itemize}
\item
سامانه باید تیمی متشکل از پشتیبانان در زمینه‌های مختلف داشته باشد (به عنوان مثال پشتیبان فنی و پشتیبان روابط عمومی).
\end{itemize}

\item رابط کاربری مناسب
\begin{itemize}
\item
سامانه باید دارای رابط کاربری مناسب باشد. به طوری که هم دارای زیبایی های بصری باشد (\lr{UI}) و هم استفاده ی کاربر از آن ساده و معلوم باشد (\lr{UX}).
\end{itemize}
\end{enumerate}

\subsection{برنامه تکرار و برنامه‌ی مرحله}
\begin{center}
\begin{longtable}{|c|c|c|c|c|}
\caption{جدول برنامه‌ی تکرار}
\endfirsthead
\endhead
\hline
نیازمندی & 
وابستگی‌ها & 
تکرار اول (۳ هفته) & 
تکرار دوم  (۳ هفته) & تکرار سوم  (۳ هفته) \\
\hline
\hline
& & & & \\ \hline
& & & & \\ \hline
& & & & \\ \hline
& & & & \\ \hline
& & & & \\ \hline
& & & & \\ \hline
& & & & \\ \hline
& & & & \\ \hline
& & & & \\ \hline
& & & & \\ \hline
& & & & \\ \hline
& & & & \\ \hline
& & & & \\ \hline
& & & & \\ \hline
& & & & \\ \hline
& & & & \\ \hline
& & & & \\ \hline
& & & & \\ \hline
& & & & \\ \hline
& & & & \\ \hline
& & & & \\ \hline
& & & & \\ \hline
& & & & \\ \hline
& & & & \\ \hline
& & & & \\ \hline
& & & & \\ \hline
& & & & \\ \hline
& & & & \\ \hline
& & & & \\ \hline
\end{longtable}
\end{center}

\chapter{مدل دامنه}
مدل دامنه، یک فرایند مفهوم‌سازی برای کمک به تیم توسعه جهت فهم دامنه‌ی کاربرد است که دارای پنج گام مختلف می‌باشد.
\begin{itemize}
\item
جمع‌آوری اطلاعات دامنه‌‌ی کاربردی
\item
طوفان فکری
\item
دسته‌بندی نتایج طوفان فکری
\item
به تصویر کشیدن مدل دامنه
\item
مرور و بازرسی مدل دامنه
\end{itemize}

\section{جمع‌آوری اطلاعات دامنه کاربردی}
مقصود اصلی از مدل‌سازی دامنه، فهم مفاهیم دامنه و چگونگی ارتباط آن‌ها با یکدیگر است، در این مرحله اعضای تیم باید مستندات یا توضیحات موجود در مورد کسب‌و‌کار را بدست آورد.
\section{طوفان فکری}
پس از جمع‌آوری اطلاعات،‌اعضای تیم در قالب ۲ جلسه به شناسایی مفاهیم مهم دامنه پرداختند، که محصول نهایی این گام که با توجه به قوانین زیر بدست آمده، فهرستی از عبارت‌های شناخته شده است.
\begin{enumerate}
\item
اسم‌‌ها یا عبارات اسمی
\item
عبارت‌های
\lr{x} از \lr{y}
یا
\lr{y x}
\item
افعال متعدی
\item
صفات، قید‌‌ها و اقلام شمارشی
\item
ارقام و اعداد و کمیت‌ها
\item
عبارت‌ها مالکیت
\item
اجزای سازنده، عبارت‌های
\underline{تشکیل شده از} و
\underline{بخشی از}
\item
عبارت‌های مربوط به دربرداشتن
\item
عبارت‌های
\lr{X}
یک
\lr{Y}
یا مفاهیم خاص کردن / تعمیم دادن است.
\end{enumerate}

\section{دسته‌بندی نتایج طوفان فکری}
در این مرحله اعضای گروه به دسته‌بندی مفاهیم دامنه پرداختند.

\section{فهرست مفاهیم مهم دامنه}
کلی کلمه و جدول بلند بلند \ref{table:domain}

\begin{longtable}{ccc}
یک کلمه
\footnotemark[2] &
دو کلمه
\footnotemark[10]&
سه کلمه 
\footnotemark \\
\end{longtable}

\begin{table}[H]
\caption{مفاهیم مهم دامنه}
\begin{longtable}{|c|c|c|}
\hline
لیست طوفان فکری &
نتیجه‌ی دسته‌بندی &
قانون \\
\hline
\end{longtable}
\label{table:domain}
\end{table}
\section{به تصویر کشیدن مدل دامنه}
یه شکل گنده :)
\section{مرور مدل دامنه}
پس از انجام همه‌ی مراحل، اعضای تیم بار دیگر به بررسی مدل دامنه می‌پردازند و در صورت وجود هرگونه اشکال آن را اصلاح می‌کنند.

\section{رعایت اصول چابکی}
کلیه مراحل مدل‌سازی دامنه با درنظر گرفتن اصول چابکی انجام شده و تیم توسعه با درنظر گرفتن کاربرد سامانه‌ی \textit{کارتاپ} و در جهت شناسایی بهتر نیازمندی‌ها سعی کرده است که با مشتری تعامل لازم را داشته باشد تا جلوی بروز هرگونه ابهام را بگیرد.

همچنین برای جلوگیری از پیچیده‌ شدن مدل دامنه در بخش طوفان فکری همه‌ی کلاس‌ها به یک باره ذکر نشده‌اند و مراحل به صورت گام‌به‌گام انجام شده چون فرایند مدل‌سازی یک فرایند تکراری‌ست و باید بازگشت‌پذیر باشد.

\chapter{طراحی معماری}

\section{فرایند طراحی معماری}	
طراحی معماری یک سیستم نرم‌افزاری یک فرایند شناختی تصمیم‌گیری به منظور تبیین ساختار کلی سیستم، زیرسیستم‌ها و ارتباط میان آنهاست و عوامل متعددی در این امر دخیل‌اند. از این عوامل می‌توان به نوع سیستم تحت توسعه و اهداف دنبال شده جهت طراحی معماری سیستم اشاره کرد. با توجه به اینکه طراحی معماری یک فرایند بازگشتی‌ست، هر سیستم متشکل از تعدادی زیرسیستم است و هر کدام از این زیرسیستم‌ها نیز از سطوح پایین‌تری تشکیل شده‌اند و تکرار فرایند بازگشتی طراحی برای هر سطح و تا پایین‌ترین سطح لازم است. پایان فرایند به عوامل گوناگونی نظیر اندازه و پیچیدگی سیستم، تجربه‌ی تیم توسعه و اهداف طراحی بستگی دارد.

\subsection{تبیین اهداف طراحی}			
ابتدا نیاز است که ملزومات اساسی و محدودیت‌های سیستم بنا بر شاخص‌های قابل توجه بررسی شوند:
\begin{enumerate}
\item
سادگی تغییر و نگهداری
\item
کاربرد قطعات تجاری
\item
کارایی سیستم
\item
قابلیت اطمینان
\item
امنیت
\item
حمل‌پذیری خطا
\item
ترمیم
\end{enumerate}

\subsection{تعیین نوع سیستم}
نوع یک سیستم، مدل‌سازی، تحلیل، طراحی، پیاده‌سازی و آزمون سیستم را بشدت تحت‌تاثیر خود قرار می‌دهد. به همین دلیل در زمان طراحی معماری نرم‌افزار، انتخاب نوع سیستم از اهمیت بالایی برخوردار است. با توجه به اهمیت تعامل بین سیستم و کنشگر برای انجام یک فرایند در \textit{کارتاپ} و اهداف طراحی معماری ذکر شده و همچنین موارد زیر، \textit{کارتاپ} یک سیستم تعاملی‌ست، معماری نرم‌افزاری \lr{N-Tire} برای آن انتخاب شده است.


\begin{enumerate}
\item		
تعامل بین سیستم و کنشگر برای انجام یک فرایند در \textit{کارتاپ} شامل دنباله‌ی ثابتی از درخواست‌های کنشگر مثل ورود،‌جستجو بین کارجویان / کارفرما‌ها و آگهی‌های شغلی پیشنهادی و درخواست (\lr{apply}) و همچنین ردخواست کارجویان می‌باشد که سیستم باید این فرایند‌ها را مدیریت کند.

\item 					
در بیشتر اوقات سیستم در هر فرایند با یک یا دو کنشگر تعامل می‌کند.

\item 					
کنشگر‌های \textit{کارتاپ} فقط شامل انسان‌ها می‌شوند.

\item 				
در همه‌ی فرایند‌ها تعامل از کنشگر شروع شده و به او ختم می‌شود.

\item 
کنشگر از سیستم،‌ خدماتی را درخواست می‌کند و سیستم به آنها پاسخ می‌دهد، به نوعی بین کنشگر و سیستم رابطه‌ی مشتری - خادم برقرار است.

\end{enumerate}

\subsection{استفاده از سبک‌ها معماری}
انواع مختلف سیستم‌‌ها، به معماری‌های متفاوت نرم‌افزار نیازمندند، بنابراین باید به توجه به سیستم در حال توسعه، سیک معماری مناسب انتخاب شود.

در سیستم‌های تعاملی، سبک معماری \lr{N-Tier} مناسب است؛ این سبک معماری، اجزای سیستم را به لایه‌های  نسبتاً مستقل با اتصال ضعیف، مرتب می‌نماید. هر لایه وظیفه و عملکرد خوش تعریف دارد و تاثیرات بر لایه‌های دیگر را کاهش می‌دهد.

در معماری \lr{N-Tier}،‌ درخواست‌ها در هر فرایند از یک لایه به لایه‌ی دیگر فرستاده می‌شود و ارسال درخواست از لایه‌ی پایین‌تر به لایه‌های بالاتر مجاز نیست.

لایه‌های این سبک معماری شامل:
\begin{enumerate}
\item 
لایه‌ی واسط گرافیکی
\item
لایه‌ی اشیای کسب‌وکار
\item 
لایه‌ی پایگاه داده
\item 
لایه‌ی ارتباط شبکه
\end{enumerate}

\subsection{زیرسیستم‌‌ها و واسط‌های سیستم}
در این گام نیازمندی‌های نرم‌افزار و اهداف طراحی آن، به زیرسیستم‌ها و مولفه‌های معماری تخصیص داده می‌شود.
\begin{enumerate}
\item \lr{Front-end Layer}:
لایه‌ی واسط گرافیکی یک گروه‌ از اشیاست که مسئول نمایش اطلاعات، منو‌ها و دکمه‌های عملیاتی به کاربر هستند، و به طول کلی در این لایه همه صفحه‌هایی که کاربر با آنها در ارتباط است، قرار دارند. مانند:
\begin{itemize}
\item 
صفحه‌ی ثبت‌نام
\item 
صفحه‌ی ورود به سامانه
\item 
صفحه‌ی ایجاد رزومه
\item 
صفحه‌ی پروفایل
\end{itemize}

\item \lr{Back-end Layer}:
این لایه مسئول پردازش و رسیدگی به درخواست‌های کاربران سامانه‌ است و تصمیمات منطقی سیستم در این لایه انجام می‌شود و یک واسط میان لایه‌های دیگر است که شامل دو زیرسیستم زیر است:
\begin{itemize}
\item \lr{Controller}:
این زیرسیستم شامل اشیای کنترل‌گر است. هر کنترل‌گر مسئول برخورد با رویداد‌های مربوط به یک مورد کاربرد مشخص است. در بیشتر موارد یک تناظر یک‌به‌یک بین مورد‌های کاربرد و اشیای کنترل‌گر برقرار است. هر شئ در زمان ارسال یک خدمت از سوی کاربر، مسئول برخورد با رویداد‌های مربوط به آن است.

\item \lr{Business}:
اشیای کسب‌وکار در این زیرسیستم وجود دارند. این بخش شامل مهم‌ترین زیرسیستم‌های سامانه می‌باشد و منطق سامانه در این بخش پیاده‌سازی می‌شود.

\end{itemize}

\item \lr{Data Layer}:
این لایه‌ از اشیایی تشکیل می‌شود که عملیات مربوط به پایگاه‌داده، مانند ذخیره‌سازی و بازیابی اشیاء را فراهم می‌آورد.

\item \lr{Network Layer}:
این لایه‌، مربوط به ارتباطات شبکه را فراهم می‌کند.

\end{enumerate}
\subsection{بازبینی طراحی معماری}
در این بخش، طراحی معماری انجام شده، بازبینی می‌شود تا از پیاده‌سازی اهداف موردنظر سیستم اطمینان حاصل شود.

\section{سبک‌ معماری و نمودار بسته}
کلی شکل :)

\section{قوانین طراحی نرم‌افزار}
بسیاری از مشکلات طراحی بر بهره‌وری و کیفیت نرم‌افزار تاثیر منفی گذاشته و هزینه‌های نگهداری نرم‌افزار را به‌شدت افزایش می‌دهند. یکی از راه‌حل‌های پیشنهاد شده برای حل اینگونه مسائل، قوانین طراحی نرم‌افزار است. استفاده‌ی صحیح آنها در طراحی نرم‌افزار، می‌تواند کیفیت نرم‌افزار را به‌شدت افزایش دهد. سامانه‌ی \textit{کارتاپ} با درنظرگرفتن این قوانین که در ادامه با جزئیات، بیان شده است، سعی کرده است که کیفیت نرم‌افزاری خود را بهبود بدهد.

\subsection{طراحی برای تغییر}
سامانه‌ی \textit{کارتاپ} بدلیل وجود یک سری رویداد، ممکن است دچار تغییراتی شود که برخی از این رویداد‌ها عبارتند از:

\begin{itemize}
\item 
وقوع اختلالات سیستمی و باگ‌های منجر به تغییر نیاز‌مندی‌های نرم‌افزاری

\item 
تغییر در قوانین و دستور‌العمل‌های محیط کسب‌وکار

\item 
تغییرات نرم‌افزاری سیستم بدلایل مختلف مانند بروزرسانی و بهبود امنیت سیستم

\item 
تغییرات سخت‌افزاری و ابزار‌های موردنیاز جهت پیاده‌سازی سیستم

\item 
ایجاد بهبود‌های موردنیاز‌ بنا بر بازخورد مشتری

\item 
تغییر زمان تحویل پروژه و بودجه اختصاص داده شده
\end{itemize}

مزیت \textit{کارتاپ} در چندلایه بودن معماری آن است و تا جایی که ممکن بوده سعی شده که لایه‌های معماری سیستم وابستگی بسیار کمی به یکدیگر و هر کدام از زیرسیستم‌ها استقلال داشته باشند. به این صورت که در صورت وقوع هرگونه تغییر اختمالی در زیرسیستم مورد نظر، سایر زیرسیستم‌ها تا حد امکان دست‌نخورده باقی خواهند ماند و این تغییرات به‌ آسانی صورت می‌گیرد.

\subsection{جداسازی دغدغه‌ها}
جداسازی دغدغه‌ها\RTLfootnote{\lr{Separation of Concerns} ایده‌ی مطرح شده ادسگر دایکسترا می‌باشد.}؛ این ایده بیان می‌کند که بجای تمرکز یکباره و همزمان به همه‌ی جنبه‌های یک مسئله، هر بار بر \textit{یکی} از جنبه‌ها و جدا از سایر آنها، تمرکز می‌شود که از انواع نمودار‌ها در این سند به همین سبب استفاده شده است. چسبندگی بالا در اثر پیاده‌سازی این کار در پروژه و تفکیک مسئولیت‌ها و دغدغه‌های گوناگون است. بنا بر تقسیم‌بندی وظایف، هر لایه دغدغه‌ی مربوط به خود را دارد؛ به عنوان مثال لایه‌ی واسط گرافیکی \textit{تنها} وظیفه‌ی نمایش اطلاعات را بر عهده دارد و لایه‌ی پایگاه‌داده، \textit{تنها} اطلاعات مربوط به کاربران را ذخیره و بازیابی می‌کند.

\subsection{پنهان‌سازی اطلاعات}
قانون پنهان‌سازی اطلاعات
\RTLfootnote{نخستین بار توسط دیوید پارناس به عنوان یک قانون طراحی معرفی گردید.}؛
مطابق این قانون، جزئیات پیاده‌سازی یک بدنه‌ی نرم‌افزاری، برای کاهش اثرات تغییر آن بر سایر قسمت‌های سیستم نرم‌افزاری، مخافظت می‌شود. \lr{N-Tier} بودن معماری سامانه‌ی \textit{کارتاپ} باعث شده که اطلاعات بصورت کلی قابل دسترسی و مشاهده نباشند و هر کدام از زیرسیستم‌های مستقل به اطلاعات مربوط به خود دسترسی داشته باشند و قابلیت دستیبابی به داده‌های موجود در سایر زیرسیستم‌ها وجود نداشته باشد.

\subsection{چسبندگی زیاد}
قانون چسبندگی زیاد توصیه می‌کند که طراحی پیمانه‌ها
\RTLfootnote{\lr{Modules}} باید طوری باشد که توابع هر پیمانه، بیشترین درجه‌ی ارتباط با مسئولیت اصلی پیمانه را داشته باشند. اعمال قانون چسبندگی زیاد در طراحی معماری به این معناست که مولفه‌ها و کلاس‌های هر زیرسیستم باید تا حدود زیادی به مسئولیت اصلی زیرسیستم مرتبط باشند. در سامانه‌ی \textit{کارتاپ} هدف کلی از وظایف محول شده به هر لایه، اجرا محقق شدن آرمان کل سیستم است و هر لایه‌ی معماری \textit{کارتاپ} توابع و کلاس‌های مربوط به خود را داراست.

\subsection{جفت‌شدگی کم}
استفاده از قانون جفت‌شدگی کم در طراحی معماری، به معنای کاهش اثرات زمان اجرا و تاثیر تغییر در سیستم بر زیرسیستم‌ها دیگر است. بخصوص، طراحی باید از متغیر‌های کنترلی دارای بیش از دو مقدار اجتناب نماید. بعلاوه، برای کاستن تاثیر تغییر، می‌توان از قوانین طراحی برای تغییر و پنهان‌سازی اطلاعات استفاده کرد و با توجه به معماری \lr{N-Tier} انتخاب شده، لایه‌های سیستم جفت‌شدگی کمی دارند و بصورت مستقل هر لایه کار مربوط به خود را انجام داده و خروجی را به لایه‌های بعدی منتقل می‌کند.

\subsection{ساده و احمقانه فرض کن}
قانون ساده و احمقانه فرض کن
\RTLfootnote{Keep It Simple Stupid (KISS)}
، طراحی‌های ساده، سرراست و قابل‌فهم را توصیه می‌کند. در این نگاه، اشیا به صورت نادان در نظر گرفته می‌شوند؛ به این معنی که هر شئ تنها توانایی انجام یک کار بخصوص را دارد و روش انجام سایر کار‌ها را نمی‌داند. تقسیم‌بندی سامانه‌ی \textit{کارتاپ} این قانون را رعایت کرده، و در هر کدام از لایه‌ها بمانند لایه‌ی واسط گرافیکی و لایه‌ی کسب‌وکار برای اجرای توابع، کلاس‌ها و اشیا به ساده‌ترین شکل ممکن تعریف شده‌اند و در نتیجه می‌توان اذعان کرد که \textit{کارتاپ} دارای اشیای احمق است.



% ----------------------------------------------------------------------------------------------------
\chapter{استنتاج مورد کاربرد‌ها از نیازمندی‌‌ها}		
در این گام، استخراج مورد کاربرد‌ها از نیازمندی‌ها صورت گرفت و در ادامه، نمودار‌‌های مورد کاربرد‌ها، جدول بازبینی و جدول تخصیص موارد کاربرد به تکرار‌‌ها ترسیم شد. کنشگران این سیستم، کاربران در نقش‌های کارجو و کارفرما می‌باشند.

\section{شناسایی مورد کاربرد‌ها}		
در این مرحله از تعداد 26 نیازمندی شناسایی شده، ۲۰ مورد کاربرد استنباط و شناسایی شد.

\section{تعیین قلمرو مورد کاربرد‌ها}		
لیست مورد کاربرد‌ها به صورت زیر است:
\begin{itemize}
\item[] \ucstep{ثبت‌نام کاربر:} 			

\tuc				
{کاربر بر روی پیوند ثبت‌نام در صفحه اصلی سایت کلیک می‌کند.}				
{کاربر در صورت موفق‌آمیز بودن ثبت‌نام، وارد پنل کاربری می‌شود.}

\item[] \ucstep{ورود به سامانه:} 			
\tuc				
{کاربر بر روی پیوند ورود به سامانه در صفحه اصلی سایت کلیک می‌کند.}				
{کاربر پنل شخصی خود را مشاهده می‌کند.}

\item[] \ucstep{خروج از سامانه:}
\tuc				
{کاربر به روی دکمه‌ی خروج در پنل کاربری کلیک می‌کند.}				
{کاربر به صفحه اصلی سایت هدایت می‌شود.}

\item[] \ucstep{بازیابی رمزعبور فراموش شده:}
\tuc				
{کابر بر روی دکمه "بازیابی رمز عبور" در صفحه ورود سایت کلیک می‌کند.}				
{کاربر پیامک حاوی رمز عبور موقت را دریافت می‌کند.}

\item[] \ucstep{مشاهده‌ی پروفایل:}
\tuc				
{کاربر بر روی آواتار در پنل کاربری خودش کلیک می‌کند.}				
{کاربر اطلاعات شخصی خود را در صفحه‌ی پروفایل مشاهده می‌کند.}

\item[] \ucstep{آپدیت اطلاعات کاربری:}
\tuc				
{کابر بر روی پیوند "تغییر اطلاعات" در قسمت نوار ابزار پنل کاربری کلیک می‌کند.}	
{کاربر پیغام "تغییر اطلاعات با موفقیت انجام شد." را مشاهده می‌کند.}

\item[] \ucstep{خرید اکانت پرمیوم:}
\tuc				
{کاربر به روی دکمه‌ی "خرید" در صفحه‌ی ارتقا اکانت کلیک می‌کند.}
{کاربر پیغام "اکانت پرمیموم با موفقیت فعال شد" را مشاهده می‌کند.}

\item[] \ucstep{ساخت رزومه:}
\tuc				
{کارجو به روی دکمه‌ی "ساخت رزومه" در صفحه‌ی پروفایل کاربری، در صحفه‌ی مربوط به کارجو، کلیک می‌کند.}
{کارجو پیغام "روزمه ساخته شد" را مشاهده می‌کند.}

\item[] \ucstep{مشاهده‌ آگهی‌های پیشنهادی:}
\tuc				
{کارجو بر روی علامت ذره‌بین (مخصوص دیدن آگهی‌ها مثل اکسپلور اینستاگرام) در صفحه‌ی اصلی کارتاپ کلیک می‌کند.}
{کارجو آگهی‌ها را مشاهده می‌کند.}

\item[] \ucstep{ذخیره ‌کردن آگهی:}
\tuc				
{کارجو بر روی دکمه‌ی "ذخیره کردن آگهی" در صفحه‌ی آگهی کلیک می‌کند.}				
{کارجو پیغام "آگهی ذخیره‌ شد." را مشاهده می‌کند.}

\item[] \ucstep{مشاهده پروفایل کارجویان:}
\tuc				
{کارفرما به روی آواتار یا نام کاربری کارجو در صفحه‌ی پیشنهادات یا جستجو کلیک می‌کند.}				
{کارفرما اطلاعات پروفایل کارجو را مشاهده می‌کند.}

\item[] \ucstep{ارسال روزمه:}
\tuc				
{کارجو بر روی دکمه "ارسال رزومه" در صفحه آگهی یک شرکت کلیک می‌کند.}
{کارجو پیغام "رزومه ارسال شد." را مشاهده می‌کند.}

\item[] \ucstep{مشاهده‌ی پروفایل شرکت‌‌ها:}
\tuc				
{کاربر بر روی پروفایل یک شرکت در صفحه معرفی سایت‌ها کلیک می‌کند.}				
{کاربر وارد صفحه مربوط به شرکت مدنظر می‌شود و اطلاعات آن را مشاهده می‌کند.}

\item[] \ucstep{ثبت آگهی توسط کارفرما:}
\tuc
{کارفرما بر روی دکمه "ثبت آگهی" در پنل کاربری کارفرما کلیک می‌کند.}
{کارفرما پیغام" آگهی با موفقیت ثبت شد" را مشاهده می‌کند.}

\item[] \ucstep{مشاهده کارجویان پیشنهادی:}
\tuc
{کارفرما بر روی گزینه "کارجویان پیشنهادی به شرکت شما" در پنل کاربری کارفرما کلیک می‌کند.}
{کارفرما اطلاعات کارجویان پیشنهادی را مشاهده می‌کند.}

\item[] \ucstep{خارج شدن از سامانه:}
\tuc				
{کاربر بر روی دکمه ی خروج در پنل کاربری کلیک می‌کند.}			
{کاربر از سامانه خارج می‌شود.}

\item[] \ucstep{نشان‌دار کردن آگهی:}
\tuc
{کارجو بر روی علامت ستاره در صفحه مربوط به آگهی مدنظر کلیک می‌کند.}
{کارجو پیغام "گهی به لیست آگهی‌های نشان‌دار افزوده شد" را مشاهده می‌کند.}

\item[] \ucstep{مشاهده وضعیت آگهی‌های درخواستی:}
\tuc		
{کارجو به روی دکمه‌ "وضعیت آگهی‌های درخواستی" در قسمت نوار ابزار پنل کاربری کارجو کلیک می‌کند.}
{کارجو لیستی از آگهی‌ها و وضعیتشان را مشاهده می‌کند.}

\item[] \ucstep{مشاهده درخواست همکاری کارفرماها:}
\tuc		
{کارجو به روی دکمه‌ "درخواست‌های همکاری" در قسمت نوار ابزار پنل کاربری کارجو کلیک می‌کند.}
{کارجو لیستی از شغل‌های پیشنهاد شده از سمت کارفرماها را مشاهده می‌کند.}

\item[] \ucstep{تغییر رمز عبور:}
\tuc				
{کاربر بر روی دکمه "تغییر رمز عبور" در پنل کاربری خودش کلیک می‌کند.}			{کاربر پیغام "رمز عبور با موفقیت تغییر کرد." را مشاهده می‌کند.}

% change to default counter ------------------------------
\renewcommand{\labelenumi}{\arabic{enumi})}
% -----------------------------------------------------------

\end{itemize}

\section{مصورسازی زمینه‌ مورد کاربرد‌ها}	
کلی شکل :)

\section{بازبینی مورد کاربرد‌ها و نمودارها}		
در این گام مورد کاربرد‌ها، نیازمندی‌ها و ارتباط میان‌ آنها مجدداً بررسی شد و در قالب جدول \ref{table:review} تدوین گردید.

\begin{sidewaystable}
\caption{جدول ردیابی موارد کاربرد}
\label{table:review}
\begin{adjustbox}{width=\textwidth}
\begin{tabular}{|c|c|c|c|c|c|c|c|c|c|c|c|c|c|c|c|c|c|c|c|c|c|}
\hline
نیازمندی‌ها &
اولویت &
\uc{01} & 
\uc{02} & 
\uc{03} & 
\uc{04} & 
\uc{05} & 
\uc{06} & 
\uc{07} & 
\uc{08} & 
\uc{09} & 
\uc{10} & 
\uc{11} & 
\uc{12} & 
\uc{13} & 
\uc{14} & 
\uc{15} & 
\uc{16} & 
\uc{17} & 
\uc{18} & 
\uc{19} & 
\uc{20} \\
\hline

\req{01} & 
&
& 
& 
& 
\zstar & 
\zstar & 
& 
& 
& 
& 
& 
& 
& 
& 
& 
& 
& 
& 
& 
& 
\\
\hline
\req{02} &
0 &
& 
& 
& 
& 
& 
& 
& 
& 
& 
& 
& 
& 
& 
& 
& 
& 
& 
& 
& 
\\
\hline
\req{03} &
0 &
& 
& 
& 
& 
& 
& 
& 
& 
& 
& 
& 
& 
& 
& 
& 
& 
& 
& 
& 
\\
\hline
\req{04} &
0 &
& 
& 
& 
& 
& 
& 
& 
& 
& 
& 
& 
& 
& 
& 
& 
& 
& 
& 
& 
\\
\hline
\req{05} &
0 &
& 
& 
& 
& 
& 
& 
& 
& 
& 
& 
& 
& 
& 
& 
& 
& 
& 
& 
& 
\\
\hline
\req{06} &
0 &
& 
& 
& 
& 
& 
& 
& 
& 
& 
& 
& 
& 
& 
& 
& 
& 
& 
& 
& 
\\
\hline
\req{07} &
0 &
& 
& 
& 
& 
& 
& 
& 
& 
& 
& 
& 
& 
& 
& 
& 
& 
& 
& 
& 
\\
\hline
\req{08} &
0 &
& 
& 
& 
& 
& 
& 
& 
& 
& 
& 
& 
& 
& 
& 
& 
& 
& 
& 
& 
\\
\hline
\req{09} &
0 &
& 
& 
& 
& 
& 
& 
& 
& 
& 
& 
& 
& 
& 
& 
& 
& 
& 
& 
& 
\\
\hline
\req{10} &
0 &
& 
& 
& 
& 
& 
& 
& 
& 
& 
& 
& 
& 
& 
& 
& 
& 
& 
& 
& 
\\
\hline
\req{11} &
0 &
& 
& 
& 
& 
& 
& 
& 
& 
& 
& 
& 
& 
& 
& 
& 
& 
& 
& 
& 
\\
\hline
\req{12} &
0 &
& 
& 
& 
& 
& 
& 
& 
& 
& 
& 
& 
& 
& 
& 
& 
& 
& 
& 
& 
\\
\hline
\req{13} &
0 &
& 
& 
& 
& 
& 
& 
& 
& 
& 
& 
& 
& 
& 
& 
& 
& 
& 
& 
& 
\\
\hline
\req{14} &
0 &
& 
& 
& 
& 
& 
& 
& 
& 
& 
& 
& 
& 
& 
& 
& 
& 
& 
& 
& 
\\
\hline
\req{15} &
0 &
& 
& 
& 
& 
& 
& 
& 
& 
& 
& 
& 
& 
& 
& 
& 
& 
& 
& 
& 
\\
\hline
\req{16} &
0 &
& 
& 
& 
& 
& 
& 
& 
& 
& 
& 
& 
& 
& 
& 
& 
& 
& 
& 
& 
\\
\hline
\req{17} &
0 &
& 
& 
& 
& 
& 
& 
& 
& 
& 
& 
& 
& 
& 
& 
& 
& 
& 
& 
& 
\\
\hline
\req{18} &
0 &
& 
& 
& 
& 
& 
& 
& 
& 
& 
& 
& 
& 
& 
& 
& 
& 
& 
& 
& 
\\
\hline

\req{19} &
0 &
& 
& 
& 
& 
& 
& 
& 
& 
& 
& 
& 
& 
& 
& 
& 
& 
& 
& 
& 
\\
\hline

\req{20} &
0 &
& 
& 
& 
& 
& 
& 
& 
& 
& 
& 
& 
& 
& 
& 
& 
& 
& 
& 
& 
\\
\hline
\req{21} &
0 &
& 
& 
& 
& 
& 
& 
& 
& 
& 
& 
& 
& 
& 
& 
& 
& 
& 
& 
& 
\\
\hline
\req{22} &
0 &
& 
& 
& 
& 
& 
& 
& 
& 
& 
& 
& 
& 
& 
& 
& 
& 
& 
& 
& 
\\
\hline
\req{23} &
0 &
& 
& 
& 
& 
& 
& 
& 
& 
& 
& 
& 
& 
& 
& 
& 
& 
& 
& 
& 
\\
\hline
\req{24} &
0 &
& 
& 
& 
& 
& 
& 
& 
& 
& 
& 
& 
& 
& 
& 
& 
& 
& 
& 
& 
\\
\hline
\req{25} &
0 &
& 
& 
& 
& 
& 
& 
& 
& 
& 
& 
& 
& 
& 
& 
& 
& 
& 
& 
& 
\\
\hline
\req{26} &
0 &
& 
& 
& 
& 
& 
& 
& 
& 
& 
& 
& 
& 
& 
& 
& 
& 
& 
& 
& 
\\
\hline
الولیت مورد کاربرد‌ها &
&
& 
& 
& 
& 
& 
& 
& 
& 
& 
& 
& 
& 
& 
& 
& 
& 
& 
& 
& 
\\
\hline

\end{tabular}
\end{adjustbox}
\end{sidewaystable}


\section{تخصیص موارد کاربرد به تکرارها}		
موارد کاربرد بر اساس اولویت آنها در هر یک از سه تکرار برنامه‌ریزی شده پخش شده‌اند که در جدول \ref{table:repeat} قابل مشاهده است.

\begin{table}
\caption{تخصیص موارد کاربرد به تکرار‌ها}
\label{table:repeat}
\begin{adjustbox}{width=\textwidth}
\begin{tabular}{|c|c|c|c|c|c|c|}

\hline
مورد کاربر‌د‌ها &
اولویت (۱-۳) &
میزان تلاش (نفر در هفته) &
وابسته به &	
تکرار ۱ (۳ هفته) &
تکرار ۲ (۳ هفته & 
تکرار ۳ (۳ هفته) \\
\hline
\uc{01} &
&
&
&
&
&
\\
\hline
\uc{02} &
&
&
&
&
&
\\
\hline
\uc{03} &
&
&
&
&
&
\\
\hline
\uc{04} &
&
&
&
&
&
\\
\hline
\uc{05} &
&
&
&
&
&
\\
\hline
\uc{06} &
&
&
&
&
&
\\
\hline
\uc{07} &
&
&
&
&
&
\\
\hline
\uc{08} &
&
&
&
&
&
\\
\hline
\uc{09} &
&
&
&
&
&
\\
\hline
\uc{10} &
&
&
&
&
&
\\
\hline
\uc{11} &
&
&
&
&
&
\\
\hline
\uc{12} &
&
&
&
&
&
\\
\hline
\uc{13} &
&
&
&
&
&
\\
\hline
\uc{14} &
&
&
&
&
&
\\
\hline
\uc{15} &
&
&
&
&
&
\\
\hline
\uc{16} &
&
&
&
&
&
\\
\hline
\uc{17} &
&
&
&
&
&
\\
\hline
\uc{18} &
&
&
&
&
&
\\
\hline
\uc{19} &
&
&
&
&
& \\
\hline

\uc{20} &
&
&
&
&
& \\
\hline
\lr{Total Effort} &
&
&
&
&
& \\
\hline

\end{tabular} 
\end{adjustbox}
\end{table}
\section{رعایت اصول چابکی}
تیم توسعه از طریق مصاحبه با کارجویان و کارفرمایان مختلف، مطالعه‌ی عملیات کسب‌وکار فعلی و همجنین جلسات اعضای گروه، توانست اطلاعات کافی و لازم جهت تدوین نیاز‌مندی‌ها و مورد کاربر‌د‌ها، بنا بر اولویت‌های مشتری را بدست آورد. در این بخش سعی شده است که مورد کاربرد‌ها در تکرار‌های منظم و با با فاصله زمانی مناسب در قالب یک تیم ۷ نفره، پیاده‌سازی شود.

\chapter{مدل‌سازی تعامل کنشگر-سیستم}
در این فصل جداول دو‌ ستونی بیانگر تعامل میان کنشگر و سیستم آمده است که شامل ورودی و خروجی کنشگر و نیز پاسخ سیستم می‌باشد.


\section{گام‌های معادل‌سازی تعامل کنشگر-سیستم}
\begin{enumerate}
\item 
ایجاد یک حدول دو ستونی

\item 
تعیین گام‌های تعامل کنشگر-سیستم

\item 
بازبینی مشخصات تعامل کنشگر-سیستم
\end{enumerate}

پس از طی مراحل فوق، جداولی که در ادامه‌ی مدل‌سازی تعامل کنشگر-سیستم آمده‌ است، رسم شده‌اند.

\section{نمودار‌های تعامل کنشگر-سیستم}
پس از مشخص شدن موارد کاربرد با مدل‌سازی تعامل کنشگر-سیستم برای برخی از مورد‌ کاربرد‌های پیچیده‌تر نمودار‌های تعامل کنشگر-سیستم برای این موارد کاربرد مشخص شده است که در شکل‌های 
قابل رؤیت است.

\begin{table}[H]
\caption{تعامل کنشگر-سیستم (یه عدد)}
% \label{table:uc0}
\begin{center}
\begin{tabular}{|r|r|}
\hline

\ucname{n}{یه اسم}
\hline

\preif{پیش‌شرط}
\hline

\actorsystem{یه نفر}
\hline

\zerostep{هعب}
\hline

\step{\tucbw: سلام} & 
\step{سلام‌تر} \\
\hline

\step{سلام} & 
\step{سلام‌تر} \\
\hline

\step{\tucew: بای} & 
\\
\hline

\end{tabular}
\end{center}
\end{table}
\bibliographystyle{plainnat-fa}
\bibliography{Ref}
\end{document}
