\documentclass{book}

\usepackage[utf8]{inputenc}
\usepackage{graphicx}
\usepackage{hyperref}
\hypersetup{
	colorlinks=true,
	linkcolor=blue
}

\usepackage{xepersian}
\settextfont{Yas}

\title{
	% \includegraphics[width=3cm, height=3cm]{20230227_204029(1).jpg} \\
	{\Huge کارتاپ}
}
\author{
	پدید‌آورندگان: \\
	سینا ربیعی \\
	سید حسین حسینی \\
	فاطمه علی‌ملکی \\
	علی قدسی مآب \\
	زهره سورانی \\
	حانیه شمس الکتابی \\
	مهدی حق‌وردی
}

\date{اسفند ۱۴۰۱}

\setcounter{secnumdepth}{3}
\setcounter{tocdepth}{3}

\begin{document}
	\maketitle
	\tableofcontents
	
	\chapter{سند نیازمندی‌ها}
		\section{مقدمه} 
			با توجه به افزایش روز افزون نرخ بیکاری در کشور ما کاریابی به صورت چشم‌گیر مورد توجه تمامی اقشار جامعه قرار گرفته است. بدین منظور ایجاد یک سامانه هدفمند برای کاهش این نرخ، سودمند است. سامانه نرم افزاری \textbf{کارتاپ}، با معرفی کارجویان به کارفرمایان و توانمندسازی افراد به منظور دریافت کار، این نیاز مهم را براورده می سازد.
			\subsection{هدف}
			\subsection{قلمرو} 
			\subsection{تعاریف، سرنام‌ها و کوته نوشته‌ها}
			\subsection{مراجع}
			\subsection{طرح کلی}
		\section{شرح کلی}
			\subsection{چشم‌انداز محصول}
				\subsubsection{واسط‌های سیستم}
					واسط‌های سیستم این مسئله را بیان می‌کند که ارتباط سامانه‌ی ما با سیستم‌های خارجی، از طریق چه واسطه‌هایی صورت می گیرد و چگونه با هم در تبادل اطلاعات مختلف هستند. به عنوان مثال:
					\begin{enumerate}
						\item 
	دسترسی به پایگاه‌داده‌ی اداره‌ی ثبت احوال برای احراز هویت کارجو‌یان، مورد نیاز است.
						\item 
	دسترسی به پایگاه‌داده‌ی اداره‌ي ثبت شرکت‌ها برای احراز هویت شرکت‌ها، مورد نیاز است.
						\item 
	از آنجایی که این پلتفرم کاربران زیادی خواهد داشت، به سرور‌های قدرتمند و سریعی جهت پاسخ به درخواست‌ها و انجام عملیات‌های لارم، نیاز داریم.
						\item 
	جهت ارتباط و اطلاع رسانی‌های مهم به کاربران از طریق پیامک، نیاز به ارتباط با سازمان‌های مخابراتی یا شرکت‌هایی‌ست که این نوع خدمات را ارائه می دهند.
					\end{enumerate}
				\subsubsection{واسط‌های کاربر}
					جهت استفاده‌ی صحیح و کارآمد کاربران از سامانه، یک سری قابلیت‌های عمومی برای همگان و یک سری قابلیت‌های خاص در پنل کاربری کاربرانِ وارد شده در حساب کاربری، وجود دارد. در نتیجه نقش کاربران تعیین کننده‌ی سطح دسترسی آن‌ها می‌باشد. 
					سطح‌ دسترسی یا نقش کاربران در این سامانه، به دو دسته تقسیم می شود:
					\begin{enumerate}
						\item کارفرما
						\item کارجو
					\end{enumerate}
				\subsubsection{واسط‌های سخت‌افزاری}
				\subsubsection{واسط‌های نرم‌افزاری} 
				\subsubsection{واسط‌های ارتباطی}
				\subsubsection{واسط‌های حافظه}
				\subsubsection{واسط‌های عملیات}
					\begin{enumerate}
						\item 
						اطلاعات بین سامانه و پایگاه داده، به صورت خودکار تبادل می شود و به صورت دستی چیزی تغییر نمی‌یابد (مگر در صورت ایجاد مشکلی خاص.)
						\item 
						برای این سامانه، نیاز به سرورهای قدرتمند و سریعی برای پردازش و ذخیره سازی داده‌ها نیاز است.
						\item 
						مراحل اعتبارسنجیِ صحت اطلاعات ورودی و فیلترهای جست‌و‌جو به صورت خودکار، توسط سامانه انجام می‌شود.
						\item 
						تمامی اطلاعات ویرایش شده یا بارگذاری شده، در همان لحظه 
						(به صورت \lr{real time} \RTLfootnote{به سیستم‌‌‌هایی گفته می‌شود که به صورت بی‌درنگ و بدون نیاز به بارگذاری (\lr{reload}) مجدد صفحه‌، اطلاعات بروزشده نمایش داده می‌شوند؛ پیام‌رسان‌ تلگرام از بهترین مثال‌های این سیستم‌هاست.})
						 در سرور‌های سامانه بروزرسانی یا بارگذاری می‌شوند.
						\item 
						در صورت استفاده‌ی بیش از حد مجاز تعداد کاربران جهت متعادل سازی سامانه، باید از طریق هدایت ترافیک به چندین سرور، دسترسی به یک دامنه را آسان‌تر و سریع‌تر کرد.
						\item 
						ارسال پیامک‌های انبوه به کاربران جهت اطلاع رسانی‌های مهم، به طور خودکار توسط سیستم‌های ارائه دهنده‌ی این نوع خدمات، انجام می‌شود.
						\item 
						سامانه باید به صورت خودکار رزومه‌های کارجویان را با درخواست‌های شغلی کارفرمایان مقایسه کند و در صورت مطابقت به طرفین پیشنهاد دهد.
						\item 
						سامانه باید مهارت‌های کارجویان را از رزومه‌های آن‌ها به طور خودکار استخراج کند.
						\item 
						احراز هویت شرکت‌ها به صورت خودکار انجام شود.
					\end{enumerate}
				\subsubsection{نیازمندی‌های سازگاری با محیط نصب}
			\subsection{کارکرد محصول}
			\subsection{مشخصات کاربر}
			\subsection{قیود}
			\subsection{قوانین کسب‌و‌کار}
			\subsection{مفروضات و وابستگی‌ها}
			
		\section{نیازمندی‌های خاص}
			\subsection{نیازمندی‌های واسط خارجی}
			\subsection{نیازمندی‌های کارکردی}
			\subsection{نیازمندی‌های کارایی}
				\begin{enumerate}
					\item 
					سامانه باید توانایی پاسخ گویی هم زمان ۱۰۰۰۰ کاربر را داشته باشد.
					\item 
					سامانه باید برای ورود کاربران از کد \lr{CAPCHA} \RTLfootnote{\lr{CAPCHA} یا همان کپچا، نرم‌افزاری آنلاین برای تولید سوالات و آزمون‌هایی‌ست که انسان براحتی قادر به پاسخ‌گویی به آنهاست ولی کامپیوتر‌ها در حال حاضر، قادر به تشخیص و پاسخ به آنها نیستند. عبارت \lr{CAPCHA} مخفف عبارت \lr{Completely Automated Public Turing Test To Tell Computers and Humans Apart} است.} استفاده کند تا از اینکه فرد وارد شده ربات نباشد، اطمینان حاصل کند.
					\item 
					سامانه باید برای ثبت نام کاربران با استفاده از کد احراز هویت، هوبت افراد را تایید نماید.
					\item 
					سیستم پیامکی سامانه باید بتواند پیامک‌ها را حداکثر ظرف ۲۰ ثانیه برای کاربران ارسال کند.
					\item 
					سامانه باید طراحی کاربرپسند داشته باشد.
					\item 
					سامانه باید قابلیت چت آنلاین را با کارشناس مربوطه برای کاربر فراهم نماید.
					\item 
					کارتاپ باید در هرگونه مواجه شدن با خطا، چه از سمت کاربر و چه از سمت سرور، خطار را با جزئیات گزارش دهد، تا نیروهای فنی این مورد را در اولین زمان ممکن بازبینی و رفع کنند.
				\end{enumerate}
			\subsection{قیود طراحی}
			\subsection{صفت‌های سیستم‌ نرم‌افزاری}
			\subsection{برنامه تکرار و برنامه‌ی مرحله}
\end{document}
