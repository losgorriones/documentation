\chapter{مدل‌سازی تعامل کنشگر-سیستم}
در این فصل جداول دو‌ ستونی بیانگر تعامل میان کنشگر و سیستم آمده است که شامل ورودی و خروجی کنشگر و نیز پاسخ سیستم می‌باشد.


\section{گام‌های معادل‌سازی تعامل کنشگر-سیستم}
\begin{enumerate}
	\item 
	ایجاد یک حدول دو ستونی
	
	\item 
	تعیین گام‌های تعامل کنشگر-سیستم
	
	\item 
	بازبینی مشخصات تعامل کنشگر-سیستم
\end{enumerate}

پس از طی مراحل فوق، جداولی که در ادامه‌ی مدل‌سازی تعامل کنشگر-سیستم آمده‌ است، رسم شده‌اند.

\section{نمودار‌های تعامل کنشگر-سیستم}\label{expandeds}
پس از مشخص شدن موارد کاربرد با مدل‌سازی تعامل کنشگر-سیستم برای برخی از مورد‌ کاربرد‌های پیچیده‌تر نمودار‌های تعامل کنشگر-سیستم برای این موارد کاربرد مشخص شده است که در جداول 
\ref{table:uc:ad}،
\ref{table:uc:apply-search}،
\ref{table:uc:see-resumes}،
\ref{table:uc:bookmark}،
\ref{table:uc:see-reqs}،
\ref{table:uc:send-resume} و
\ref{table:uc:signup}
قابل رؤیت است.

\begin{table}[H]
	\caption{تعامل کنشگر-سیستم \arabic{table} (ثبت آگهی توسط کارفرما)}
	\label{table:uc:ad}
	\begin{center}
		\begin{tabular}{|p{0.5\textwidth}|p{0.5\textwidth}|}
			\hline
			
			\ucname{14}{ثبت آگهی توسط کارفرما}
			\hline
			
			\preif{کارفرما وارد سیستم شده باشد.}
			\hline
			
			\actorsystem{کارفرما}
			\hline
			
			\zerostep{{\small سیستم پنل کاربری کارفرما را نمایش بدهد.}}
			\hline
			
			\step{{\small \textbf{\tucbw}: کارفرما بر روی دکمه‌ی \say{ثبت آگهی} در پنل کاربری کلیک می‌کند.}} & 
			\step{{\small سیستم فرم مربوط به ایجاد آگهی‌ را نشان می‌دهد.}} \\
			\hline
			
			\step{{\small کارفرما اطلاعات مروبط به آگهی را در فرم وارد کرده و بر روی دکمه‌ی \say{ثبت آگهی} کلیک می‌کند.}} &
			
			\step{{\small سیستم صفحه‌ی بررسی و پرداخت صورت حساب آگهی را نشان می‌دهد.}} \\ \hline
			
			\step{{\small کارفرما بر روی دکمه‌ی \say{پرداخت از طریق درگاه بانکی} کلیک می‌کند.}} &
			
			% --------------------------------------------------------
			\renewcommand{\labelenumi}{\alph{enumi})}
			% --------------------------------------------------------
			
			\step{{\small سیستم اطلاعات تراکنش را به درگاه بانکی ارسال می‌کند و نتیجه تراکنش را دریافت  می‌کند:
					\begin{enumerate}
						\item 
						اگر تراکنش موفقت‌آمیز بود، سیستم به صفحه‌ی پنل کاربری بازگشته و پیغام \say{آگهی با موفقیت ثبت شد.} را نمایش می‌دهد
						\item 
						اگر تراکنش ناموفق بود، سیستم به صفحه‌ی پرداخت بازگشته و پیغام \say{پرداخت ناموفق بود، آگهی ثبت نشد.} را نمایش می‌دهد.
					\end{enumerate}
			}} \\ \hline
			
			% --------------------------------------------------------
			\renewcommand{\labelenumi}{\alph{enumi})}
			% --------------------------------------------------------
			
			\step{{\small \textbf{\tucew}: کارفرما پیغام 
					\begin{enumerate}
						\item 
						آگهی با موفقیت ثبت شد.
						\item 
						پرداخت ناموفق بود، آگهی ثبت نشد.
				\end{enumerate}}  را مشاهده می‌کند.} &
			\\
			\hline
		\end{tabular}
	\end{center}
\end{table}
% --------------------------------------------------------
\renewcommand{\labelenumi}{\arabic{enumi})}
% --------------------------------------------------------

\setcounter{UseCaseCounter}{0}
\begin{table}[H]
	\caption{تعامل کنشگر-سیستم \arabic{table} (‌جستجوی آگهی‌ها)}
	\label{table:uc:apply-search}
	\begin{center}
		\begin{tabular}{|p{0.5\textwidth}|p{0.5\textwidth}|}
			\hline
			
			\ucname{23}{جستجوی آگهی‌ها}
			\hline
			
			\preif{کاربر وارد سیستم شده باشد.}
			\hline
			
			\actorsystem{کاربر}
			\hline
			
			\zerostep{{\small سیستم صفحه‌‌ی اصلی سایت را نشان می‌دهد که این صفحه، نوار جستجو را هم داراست.}}
			\hline
			
			\step{{\small \textbf{\tucbw}: کاربر روی نوار جستجو در صفحه‌ی اصلی کلیک می‌کند.}} & 
			\\ \hline
			
			\step{کاربر عبارت جستجو را در نوار جستجو وارد می‌کند.} &
			\step{{\small سیستم عبارت را در پایگاه داده جستجو می‌کند.}} \\
			\hline
			
			\step{{\small کاربر یک صفحه‌ی \lr{preload} را مشاهده می‌کند و اندکی منتظر نتایج می‌ماند.}} &
			\step{{\small سیستم صفحه‌ی نتایج جستجو را با نتایج پر می‌کند.}} \\
			\hline
			
			\step{{\small \textbf{\tucew}: کاربر نتیجه‌ی جستجو را مشاهده می‌کند.}} & 
			\\
			\hline
			
		\end{tabular}
	\end{center}
\end{table}

\setcounter{UseCaseCounter}{0}
\begin{table}[H]
	\caption{تعامل کنشگر-سیستم \arabic{table} (مشاهده‌ی رزومه‌ها)}
	\label{table:uc:see-resumes}
	\begin{center}
		\begin{tabular}{|p{0.5\textwidth}|p{0.5\textwidth}|}
			\hline
			
			\ucname{25}{مشاهده‌ی رزومه‌ها}
			\hline
			
			\preif{کارفرما وارد سیستم شده باشد.}
			\hline
			
			\actorsystem{کارفرما}
			\hline
			
			\zerostep{{\small سیستم پروفایل یک کارجو را نشان می‌دهد.}}
			\hline
			
			\step{\textbf{\tucbw}: کارفرما بر روی دکمه‌ی \say{رزومه} در صفحه‌ی پروفایل کارجوی مدنظر کلیک می‌کند.} & 
			
			\renewcommand{\labelenumi}{\alph{enumi})}
			\step{{\small سیستم رزومه‌ي کارجو را از پایگاه داده خوانده و  
					\begin{enumerate}
						\item اگر رزومه‌ای وجود داشت، آن را به کارفرما نشان می‌دهد.
						\item
						اگر رزومه وجود نداشت پیغام \say{عدم وجود رزومه} را نشان می‌دهد.
					\end{enumerate}
			}} \\
			\hline
			\renewcommand{\labelenumi}{\arabic{enumi})}
			\step{{\small \textbf{\tucew}: کارفرما یا رزومه را مشاهده کرده یا پیغام \say{عدم پیغام رزومه} را می‌بیند.}} & 
			\\
			\hline
			
		\end{tabular}
	\end{center}
\end{table}

\setcounter{UseCaseCounter}{0}
\begin{table}[H]
	\caption{تعامل کنشگر-سیستم \arabic{table} (نشان‌دار‌ کردن آگهی)}
	\label{table:uc:bookmark}
	\begin{center}
		\begin{tabular}{|p{0.5\textwidth}|p{0.5\textwidth}|}
			\hline
			
			\ucname{17}{نشان‌دار‌ کردن آگهی}
			\hline
			
			\preif{کارجو وارد سیستم شده باشد.}
			\hline
			
			\actorsystem{کارجو}
			\hline
			
			\zerostep{{\small سیستم صفحه‌ی یک آگهی را نشان می‌دهد.}}
			\hline
			
			\step{{\small \textbf{\tucbw}: کارجو بر روی علامت ستاره در صفحه‌ی مربوط به آگهی مدنظر کلیک می‌کند}} & 
			
			% --------------------------------------------------------
			\renewcommand{\labelenumi}{\alph{enumi})}
			% --------------------------------------------------------
			
			\step{{\small \begin{enumerate}
						\item اگر آگهی جزو نشان‌دار‌ها بود، از نشان‌دار‌ها حذف شود و پیغام \say{آگهی از نشان‌دارها حذف شد} را نشان دهد.
						\item در غیر این صورت آگهی را به آگهی‌های نشان‌دار اضافه و پیغام \say{آگهی به آگهی‌های نشان‌دار افزوده شد.} را نشان دهد.
			\end{enumerate}}} \\
			\hline
			% --------------------------------------------------------
			\renewcommand{\labelenumi}{\arabic{enumi})}
			% --------------------------------------------------------
			
			\step{{\small \textbf{\tucew}: کارجو پیغام \say{آگهی به لیست آگهی‌های نشان‌دار افزوده شد} یا \say{آگهی‌ از نشان‌دار‌ها حذف شد.} را مشاهده می‌کند.}} & 
			\\
			\hline
			
		\end{tabular}
	\end{center}
\end{table}

\setcounter{UseCaseCounter}{0}
\begin{table}[H]
	\caption{تعامل کنشگر-سیستم \arabic{table} (مشاهده‌ی وضعیت آگهی‌های درخواستی)}
	\label{table:uc:see-reqs}
	\begin{center}
		\begin{tabular}{|p{0.5\textwidth}|p{0.5\textwidth}|}
			\hline
			
			\ucname{18}{مشاهده‌ی وضعیت آگهی‌های درخواستی}
			\hline
			
			\preif{کارجو وارد سیستم شده باشد.}
			\hline
			
			\actorsystem{کارجو}
			\hline
			
			\zerostep{{\small سیستم پنل‌کاربری کارجو را نشان‌ می‌دهد.}}
			\hline
			
			\step{{\small\textbf{\tucbw}: کارجو به روی دکمه‌ی \say{وضعیت‌ آگهی‌های درخواستی} در قسمت نوار ابزار پنل کاربری کارجو، کلیک می‌کند.}} & 
			
			\step{{\small سیستم آگهی‌های درخواست داده شده توسط کارجو و وضعیت‌ آنها را از پایگاه داده دریافت کرده و به کاربر نشان می‌دهد}} \\
			\hline
			
			\step{{\small\textbf{\tucew}: کارجو لیستی از آگهی‌ها و وضعیتشان را مشاهده می‌کند.}} & 
			\\
			\hline
			
		\end{tabular}
	\end{center}
\end{table}

\setcounter{UseCaseCounter}{0}
\begin{table}[H]
	\caption{تعامل کنشگر-سیستم \arabic{table} (ارسال رزومه)}
	\label{table:uc:send-resume}
	\begin{center}
		\begin{tabular}{|p{0.5\textwidth}|p{0.5\textwidth}|}
			\hline
			
			\ucname{12}{ارسال رزومه}
			\hline
			
			\preif{کاربر وارد شده باشد.}
			\hline
			
			\actorsystem{کارجو}
			\hline
			
			\zerostep{{\small سیستم صفحه‌ی مربوط به آگهی مدنظر را نمایش دهد.}}
			\hline
			
			\step{{\small \textbf{\tucbw}: کارجو بر روی دکمه‌ی \say{ارسال رزومه} در صفحه‌ی آگهی یک شرکت کلیک می‌کند.}} & 
			\step{{\small سیستم کارجو را به صفحه‌ی بارگذاری رزومه هدایت می‌کند.}} \\
			\hline
	
			\step{{\small کارجو رزومه‌ی خود را بارگذاری می‌کند و به روی دکمه‌ی \say{ارسال} کلیک می‌کند.}} &

			% --------------------------------------------------------
			\renewcommand{\labelenumi}{\alph{enumi})}
			% --------------------------------------------------------

\step{{\small 
سیستم فایل آپلود شده را بررسی می‌کند:
\begin{enumerate}
	\item 
	اگر فایل آپلود شده \lr{PDF} بود، آنرا برای کارفرما ارسال می‌کند و  پیغام \say{رزومه ارسال شد.} را نمایش می‌دهد.
	\item 
	در غیر این صورت، کارجو را به صفحه‌ی ارسال رزومه هدایت می‌کند و پیغام \say{فرمت فایل ارسالی درست نیست، لطفا مجدداً تلاش کنید.} را نمایش می‌دهد.
\end{enumerate}
}} \\ \hline
			% --------------------------------------------------------
			\renewcommand{\labelenumi}{\arabic{enumi})}
			% --------------------------------------------------------

			\step{{\small \textbf{\tucew}: کارجو پیغام \say{پیغام رزومه ارسال شد.} یا پیغام \say{فرمت فایل ارسالی درست نیست، لطفا مجدداً تلاش کنید.} را مشاهده می‌کند.}} & 
			\\
			\hline
			
		\end{tabular}
	\end{center}
\end{table}

\setcounter{UseCaseCounter}{0}
\begin{table}[H]
	\caption{تعامل کنشگر-سیستم \arabic{table} (ثبت‌نام کاربر)}
	\label{table:uc:signup}
	\begin{center}
		\begin{tabular}{|p{0.5\textwidth}|p{0.5\textwidth}|}
			\hline
			
			\ucname{1}{ثبت‌نام کاربر}
			\hline
			
			\preif{}
			\hline
			
			\actorsystem{کاربر}
			\hline
			
			\zerostep{{\small سیستم صفحه‌ی اصلی کارتاپ که حاوی دکمه‌ی \say{ثبت‌نام} است را نشان بدهد}}
			\hline			
\step{{\small \textbf{\tucbw}: کاربر بر روی دکمه ثبت‌نام در صفحه‌ی اصلی سایت کلیک می‌کند.}} & 

\step{{\small سیستم صفحه‌ی ثبت‌نام را به کاربر نشان می‌دهد.}}
\\ \hline

\step{{\small کاربر بین گزینه‌ی \say{کارجو} و \say{کارفرما} یکی را انتخاب می‌کند.}} & 

% --------------------------------------------------------
\renewcommand{\labelenumi}{\alph{enumi})}
% --------------------------------------------------------

\step{{\small پاسخ سیستم طبق انتخاب کاربر، این موارد می‌باشد:
\begin{enumerate}
	\item
	اگر کارجو انتخاب شد، سیستم کاربر را به صفحه‌ی ثبت‌نام کارجو هدایت می‌کند. 
	\item
		اگر کارفرما انتخاب شد، سیستم کاربر را به صفحه‌ی ثبت‌نام کارفرما هدایت می‌کند.
\end{enumerate}
}}
\\ \hline
% --------------------------------------------------------
\renewcommand{\labelenumi}{\arabic{enumi})}
% --------------------------------------------------------


\step{{\small کاربر اطلاعات مورد نیاز را وارد می‌کند و دکمه‌ی \say{ثبت‌نام} را کلیک می‌کند.}} & 

% --------------------------------------------------------
\renewcommand{\labelenumi}{\alph{enumi})}
% --------------------------------------------------------

\step{{\small سیستم اطلاعات وارد شده را بررسی می‌کند
\begin{enumerate}
	\item 
	اگر ثبت‌نام موفقیت‌آمیز بود، پیغام \say{ثبت‌نام موفقیت‌آمیز بود.} را نشان می‌دهد کاربر را به پنل کاربری او هدایت می‌کند.
	\item 
	در غیر این صورت  پیغام \say{ثبت‌نام انجام نشد.} را نشان می‌دهد و کاربر را به صفحه‌ی ثبت‌‌نام هدایت می‌کند.
\end{enumerate}
}}
\\ \hline
% --------------------------------------------------------
\renewcommand{\labelenumi}{\arabic{enumi})}
% --------------------------------------------------------

\step{{\small \textbf{\tucew}: کاربر در صورت موفقیت‌آمیز بودن ثبت‌نام وارد پنل کاربری می‌شود، در غیر این صورت به صفحه‌ی اصلی سایت هدایت می‌شود.}} & 
\\ \hline

		\end{tabular}
	\end{center}
\end{table}
